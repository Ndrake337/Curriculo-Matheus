\documentclass{article}
\usepackage[cm]{fullpage}
\usepackage{color}
\usepackage{hyperref}
\usepackage{parskip}

\hypersetup{breaklinks=true,%
pagecolor=white,%
colorlinks=true,%
linkcolor=cyan,%
urlcolor=MyDarkBlue}

\definecolor{MyDarkBlue}{rgb}{0,0.0,0.45}

%%%%%%%%%%%%%%%%%%%%%%%%%%
% Formatting parameters  %
%%%%%%%%%%%%%%%%%%%%%%%%%%

\newlength{\tabin}
\setlength{\tabin}{1em}
\newlength{\secsep}
\setlength{\secsep}{0.1cm}

\setlength{\parindent}{0in}
\setlength{\parskip}{0in}
\setlength{\itemsep}{0in}
\setlength{\topsep}{0in}
\setlength{\tabcolsep}{0in}

\definecolor{contactgray}{gray}{0.3}
\pagestyle{empty}

%%%%%%%%%%%%%%%%%%%%%%%%%%
%  Template Definitions  %
%%%%%%%%%%%%%%%%%%%%%%%%%%

\newcommand{\lineunder}{\vspace*{-8pt} \\ \hspace*{-6pt} \hrulefill \\ \vspace*{-15pt}}
\newcommand{\name}[1]{\begin{center}\textsc{\Huge#1}\\\end{center}}
\newcommand{\program}[1]{\begin{center}\textsc{#1}\end{center}}
\newcommand{\contact}[1]{\begin{center}\color{contactgray}{\small#1}\end{center}}

\newenvironment{tabbedsection}[1]{
  \begin{list}{}{
      \setlength{\itemsep}{0pt}
      \setlength{\labelsep}{0pt}
      \setlength{\labelwidth}{0pt}
      \setlength{\leftmargin}{\tabin}
      \setlength{\rightmargin}{\tabin}
      \setlength{\listparindent}{0pt}
      \setlength{\parsep}{0pt}
      \setlength{\parskip}{0pt}
      \setlength{\partopsep}{0pt}
      \setlength{\topsep}{#1}
    }
  \item[]
}{\end{list}}

\newenvironment{nospacetabbing}{
    \begin{tabbing}
}{\end{tabbing}\vspace{-1.2em}}

\newenvironment{resume_header}{}{\vspace{0pt}}


\newenvironment{resume_section}[1]{
  \filbreak
  \vspace{2\secsep}
  \textsc{\large#1}
  \lineunder
  \begin{tabbedsection}{\secsep}
}{\end{tabbedsection}}

\newenvironment{resume_subsection}[2][]{
  \textbf{#2} \hfill {\footnotesize #1} \hspace{2em}
  \begin{tabbedsection}{0.5\secsep}
}{\end{tabbedsection}}

\newenvironment{subitems}{
  \renewcommand{\labelitemi}{-}
  \begin{itemize}
      \setlength{\labelsep}{1em}
}{\end{itemize}}

\newenvironment{resume_employer}[4]{
  \vspace{\secsep}
  \textbf{#1} \\ 
  \indent {\small #2} \hfill {\footnotesize#3 (#4)}
  \begin{tabbedsection}{0pt}
  \begin{subitems}
}{\end{subitems}\end{tabbedsection}}


%%%%%%%%%%%%%%%%%%%%%%%%%%
%     Start Document     %
%%%%%%%%%%%%%%%%%%%%%%%%%%

\begin{document}

\begin{resume_header}
\name{Matheus Felipe}
\program{Engenheiro da Computação -- Técnico em Multimidia}
\contact{\href{mailto:matheus.felipe337@gmail.com}{matheus.felipe337@gmaill.com} \hspace{2cm} +55-11-996315436 \hspace{2cm} \href{https://www.github.com/Ndrake337}{Github.com/Ndrake337}}
\end{resume_header}

\begin{resume_section}{Sobre Mim}
  \begin{nospacetabbing}

  \textbf{Hard Skills}  \= C/C++; Python; Javacript/Typescript; Node.JS; Linux; Git/GitHub; SQL Server; Alteryx; Java; \\* \> Tableau; React/React Native; Next.js; Express; Fastify; \\\\*
  \textbf{Soft Skills}  \> Metodologia ágil; Comunicação; Clean Code;\\*
  \textbf{Idiomas}   \> Inglês Intermediário; Espanhol Basico\\*
  \textbf{Design} \> Adobe Photoshop; Adobe After Effects; Adobe Illustrator; Figma;\\*
  \textbf{Interesses} \>  Desenvolvimento Back-end; Ciencia de Dados; Desnvolvimento Front-End (Mobile e Web);\\*
  \end{nospacetabbing}

\end{resume_section}

\begin{resume_section}{Experiencia Profissional}
  \begin{resume_employer}{Itaú Unibanco}{Analista de Engenharia de TI Junior}{São Paulo, SP}{Fev/21 - Fev/22}
    \item Criação de Scripts automatizado em VBA. 
    \item Criação de Scripts Python e Alteryx para extração de dados e Preparação de dados.
    \item Manutenção de bases de dados nos ambientes Hadoop/SQL/SAS/ATHENA.
    \item Construção de Paineis para visualização de dados em Tableau. \\*
  \end{resume_employer}
  
  \begin{resume_employer}{Itaú Unibanco}{Estagiário em Decision Science Crédito}{São Paulo, SP}{Fev/21 - Fev/22}
    \item Criação e manutenção de Scripts SAS para geração de Bases e indicadores de crédito.
    \item Criação e manutenção de fluxos Alteryx visando CDP (AWS).
    \item Construção de Paineis para visualização de dados em Tableau.\\*
  \end{resume_employer}
  
  \begin{resume_employer}{Itgoal}{Estagiário em Back-end}{São Paulo, SP}{Mar/20 - Nov/20}
    \item	Desenvolvimento de automações e Scripts para sistemas Zoho.
    \item	Construção de Dashboards, utilizando SQL em Zoho Analytics.
    \item	Integração de APIs.
    \item	Helpdesk para Suporte.
  \end{resume_employer}
\end{resume_section}


\begin{resume_section}{Projetos Pessoais}
  \begin{resume_subsection}[(Julho-2022)]{Projeto Integrador - Estoque Inteligente}
    \begin{subitems}
      \item Idealizado junto a um grupo de 4 pessoas, o projeto visa criar uma prateleira inteligente capaz de detereminar a quantidade de items sobre sua base baseado no peso médio do item
      \item Desenvolvimento dos sistemas Web (Front-End e Back-End)
      \item Hospedagem de aplicação Web desenvolvida em React hospedada em github no AWS usando Pipelines
      \item Deploy de API criada em Flask em ambiente produtivo via Heroku
    \end{subitems}
  \end{resume_subsection}
\end{resume_section}

\begin{resume_section}{Educação}
  \begin{resume_subsection}[Bacharelado em Engenharia da Computação, (2019--Present)]{Centro Universitário SENAC}
    \begin{subitems}
      \item Aplicação pratica de Algoritimos de Computação em C.
      \item Utilização de ambientes Linux em maquinas virtuais.
      \item	Práticas em construção de dispositivos IoT integrados na nuvem com Microcontroladores.
      \item Conhecimento em circuitos integrados.
      \\*
    \end{subitems}
  \end{resume_subsection}
  
  \begin{resume_subsection}[Técnico em Multimidia, (2017--2018)]{ETEC Jornalista Roberto Marinho}
    \begin{subitems}
      \item Edição de Imagens, Som, Animação e Video
      \item Tecnicas de UX/UI para Web
     \end{subitems}
  \end{resume_subsection}
\end{resume_section}

\end{document}